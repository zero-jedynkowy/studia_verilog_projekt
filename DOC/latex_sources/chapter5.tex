\noindent
Na ilustracji nr. 2 przedstawiono wykonanie dwóch operacji: odejmowanie liczb A i B oraz zmianę bitu w argumencie A oznaczonego indeksem B (moduły: \emph{mod1} i \emph{mod3}). Przy pierwszej operacji na początku wynik jest niezdefiniowany i flaga błędu ustawiona na 1; przepełnienie wartości. W kolejnej operacji w argumencie B został zmieniony na indeksie B: B równe jest 3, więc bit nr. 3 (liczony od zera) w A został zmieniony na 0. Przy operacji ustawiania bitu poprzez B widać ustawienie flagi 0b0100 która oznacza że wynik posiada parzystą liczbę jedynek (co też jest widoczne na wyjściu result).

\begin{figure}[h!]
	\centering
	\includegraphics[width=1\linewidth]{img1}
	\caption{Widok wykonania testbenchu RTL i oryginalnych plikow w GTKWave}
	\label{fig:img1}
\end{figure}

