Układ realizuje 4 operacje (4 podmoduły):

\subsection{Podmoduł \emph{mod1}:}
Odejmowanie argumentów (A -  B); jeśli operacja nie może zostać wykonana, jednostka zgłasza błąd, a wyjście jest niezdefiniowane. Kod modułu \textbf{\emph{i\_oper}}: 00.

\subsubsection*{Wejścia}
\begin{itemize}
	\item \emph{\textbf{i\_argA}} - m-bitowe wejście,
	\item \emph{\textbf{i\_argB}} - m-bitowe wejście,
\end{itemize}
\subsubsection*{Wyjścia}
\begin{itemize} 
	\item \emph{\textbf{o\_result}} - m-bitowe wyjście,
	\item \emph{\textbf{o\_status}} - 4 bitowe wyjście.
\end{itemize}

\subsection{Podmoduł \emph{mod2}:}
Porównanie argumentów (A < B); jeśli warunek jest spełniony to wynikiem jest liczba 1, w przeciwnym wypadku wynikiem jest 0. Kod modułu \textbf{\emph{i\_oper}}: 01.

\subsubsection*{Wejścia}
\begin{itemize}
	\item \emph{\textbf{i\_argA}} - m-bitowe wejście,
	\item \emph{\textbf{i\_argB}} - m-bitowe wejście,
\end{itemize}
\subsubsection*{Wyjścia}
\begin{itemize}
	\item \emph{\textbf{o\_result}} - m-bitowe wyjście,
	\item \emph{\textbf{o\_status}} - 4 bitowe wyjście.
\end{itemize}

\subsection{Podmoduł \emph{mod3}:}
Ustawienie bitu w argumencie A na wartość 0; numer bitu jest określony w argumencie B; zgłoszenie błędu jeśli wartość B jest ujemna lub przekrasza liczbę bitów argumentu A. Kod modułu \textbf{\emph{i\_oper}}: 10.

\subsubsection*{Wejścia}
\begin{itemize}
	\item \emph{\textbf{i\_argA}} - m-bitowe wejście,
	\item \emph{\textbf{i\_argB}} - m-bitowe wejście,
\end{itemize}
\subsubsection*{Wyjścia}
\begin{itemize}
	\item \emph{\textbf{o\_result}} - m-bitowe wyjście,
	\item \emph{\textbf{o\_status}} - 4 bitowe wyjście.
\end{itemize}

\subsection{Podmoduł \emph{mod4}:}
Konwersja argumentu A z kodu ZNAK-MODUŁ na U2; jeśli konwersja nie może zostać wykonana - zgłaszany jest błąd a wynik jest nieokreślony. Kod modułu \textbf{\emph{i\_oper}}: 11.

\subsubsection*{Wejścia}
\begin{itemize}
	\item \emph{\textbf{i\_argA}} - m-bitowe wejście,
\end{itemize}
\subsubsection*{Wyjścia}
\begin{itemize}
	\item \emph{\textbf{o\_result}} - m-bitowe wyjście,
	\item \emph{\textbf{o\_status}} - 4 bitowe wyjście.
\end{itemize}