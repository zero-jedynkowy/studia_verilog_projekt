Układ realizuje 4 operacje (4 podmoduły):

\subsection{Podmoduł \emph{mod1}:}
Odejmowanie argumentów (A - B); jeśli operacja nie może zostać wykonana, jednostka zgłasza błąd, a wyjście jest niezdefiniowane.

\subsubsection*{Wejścia}
\begin{itemize}
	\item i\_argA - m-bitowe wejście,
	\item i\_argB - m-bitowe wejście,
\end{itemize}
\subsubsection*{Wyjścia}
\begin{itemize}
	\item o\_result - m-bitowe wyjście,
	\item o\_status - 4 bitowe wyjście.
\end{itemize}

\subsection{Podmoduł \emph{mod2}:}
Porównanie argumentów (A < B); jeśli warunek jest spełniony to wynikiem jest liczba 1, w przeciwnym wypadku wynikiem jest 0,

\subsubsection*{Wejścia}
\begin{itemize}
	\item i\_argA - m-bitowe wejście,
	\item i\_argB - m-bitowe wejście,
\end{itemize}
\subsubsection*{Wyjścia}
\begin{itemize}
	\item o\_result - m-bitowe wyjście,
	\item o\_status - 4 bitowe wyjście.
\end{itemize}

\subsection{Podmoduł \emph{mod3}:}
Ustawienie bitu w argumencie A na wartość 0; numer bitu jest określony w argumencie B; zgłoszenie błędu jeśli wartość B jest ujemna lub przekrasza liczbę bitów argumentu A,

\subsubsection*{Wejścia}
\begin{itemize}
	\item i\_argA - m-bitowe wejście,
	\item i\_argB - m-bitowe wejście,
\end{itemize}
\subsubsection*{Wyjścia}
\begin{itemize}
	\item o\_result - m-bitowe wyjście,
	\item o\_status - 4 bitowe wyjście.
\end{itemize}

\subsection{Podmoduł \emph{mod4}:}
Konwersja argumentu A z kodu ZNAK-MODUŁ na U2; jeśli konwersja nie może zostać wykonana - zgłaszany jest błąd a wynik jest nieokreślony.

\subsubsection*{Wejścia}
\begin{itemize}
	\item i\_argA - m-bitowe wejście,
\end{itemize}
\subsubsection*{Wyjścia}
\begin{itemize}
	\item o\_result - m-bitowe wyjście,
	\item o\_status - 4 bitowe wyjście.
\end{itemize}