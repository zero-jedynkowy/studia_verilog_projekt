\noindent
Założenia systemowe: Linux (testowano na Ubuntu (\textbf{\emph{makefile}}) i Windows, (gdzie komendy z \textbf{\emph{makefile}} są wpisywane ręcznie).

\bigskip

\noindent
Założenia dotyczące plików:
\begin{itemize}
	\item \textbf{\emph{makefile}}, znajdujący się w folderze \textbf{\emph{WORK}},
	\item \textbf{\emph{run.ys}}, znajdujący się w folderze \textbf{\emph{WORK}}),
	\item \textbf{\emph{exe\_unit\_w1.sv}}, znajdujący się w folderze \textbf{\emph{MODEL}},
	\item \textbf{\emph{otherModules.sv}}, znajdujący się w folderze \textbf{\emph{MODEL}},
	\item \textbf{\emph{testbench.sv}}, znajdujący się w folderze \textbf{\emph{MODEL}},
\end{itemize}

\noindent
znajdują się w jednym folderze. Aby zsyntezować projekt, wykonać testbench i wyświetlić przebiegi wystarczy użyć komendy:


\begin{lstlisting}[language=bash, style=rm]
		$ ./makefile
\end{lstlisting}

\bigskip

\noindent
Testowano przy użyciu: Icarus Verilog w wersji 11 i 12 oraz Yosys w wersji 0.23.